%\documentclass[12pt,a4paper]{amsart}
\documentclass[10pt,a4paper]{article}
\ifx\pdfpageheight\undefined\PassOptionsToPackage{dvips}{graphicx}\else%
\PassOptionsToPackage{pdftex}{graphicx}
\PassOptionsToPackage{pdftex}{color}
\fi

\usepackage{diagrams}
\usepackage[all]{xy}
\usepackage{url}
\usepackage{verbatim}
\usepackage{latexsym}
\usepackage{amssymb,amstext,amsmath}

\usepackage{float}
\floatstyle{boxed}
\restylefloat{figure}

\newcommand\myfrac[2]{
 \begin{array}{c}
 #1 \\
 \hline \hline
 #2
\end{array}}

\newcommand{\Ref}{{\sf Ref}}
\newcommand{\JJ}{{\sf J}}
\newcommand{\HH}{{\sf H}}
\newcommand{\inh}{\mathsf{inh}}

\newcommand{\mkbox}[1]{\ensuremath{#1}}
\newcommand{\Id}{{\sf Id}}

\newcommand{\Idd}{{\sf Id}}
\newcommand{\CC}{{\mathcal C}}
\newcommand{\subst}{{\sf subst}}
\newcommand{\res}{{\sf res}}
\newcommand{\Int}{{\bf I}}
\newcommand{\sem}[1]{\langle #1\rangle}
\def\SET{{U}}

\newcommand{\Sph}{{\sf S}^1}
\newcommand{\pair}[1]{{\langle #1 \rangle}}
\newcommand{\Prod}[2]{\displaystyle\prod _{#1}~#2}
\newcommand{\Sum}[2]{\displaystyle\sum _{#1}~#2}
\newcommand{\gothic}{\mathfrak}
\newcommand{\omicron}{*}
\newcommand{\gP}{{\gothic p}}
\newcommand{\lift}[1]{\tilde{#1}}
\newcommand{\gM}{{\gothic M}}
\newcommand{\gN}{{\gothic N}}
\newcommand{\rats}{\mathbb{Q}}
\newcommand{\ints}{\mathbb{Z}}

\usepackage{epsf}
\usepackage{epsfig}
% \usepackage{isolatin1}
\usepackage{a4wide}
\usepackage{verbatim}
\usepackage{proof}
\usepackage{latexsym}
\usepackage{amssymb}
% \usepackage{stmaryrd}
\newcommand{\abs}[2]{\lambda #1 . #2}            % abstraction of #1 in #2
\usepackage{mytheorems}
%\newtheorem{proposition}[theorem]{Proposition}
\newcommand{\GG}{{\sf f}}
\newcommand{\ext}{{\sf ext}}

%\documentstyle{article}
\newcommand{\IF}[3]{{{\sf if}~#1~{\sf then}~#2~{\sf else}~#3}}
\newcommand{\TT}{{1}}
\newcommand{\FF}{{0}}
\newcommand{\lfpi}[3]{(\Pi #1{:}#2)#3}
\newcommand{\lflam}[2]{\lambda #1.#2}
\newcommand{\HA}{{\sf HA}}
\newcommand{\AC}{{\sf AC}}
\newcommand{\HAw}{\hbox{\sf{HA}$^{\omega}$}}
\newcommand{\EM}{\hbox{\sf{EM}}}
\newcommand{\DC}{\hbox{\sf{DC}}}
\newcommand{\BB}{\hbox{\sf{B}}}

\def\NN{\hbox{\sf N}}
\def\Type{\hbox{\sf Type}}
\def\Box{\hbox{\sf B}}
\def\PER{\hbox{\sf PER}}
\def\FUN{\Pi}
\def\ELEM{\hbox{\sf El}}
\def\GET{\hbox{\sf get}}
\def\TP{\hbox{\sf TP}}
\def\N0{\hbox{\sf N}_0}
\def\ZERO{\hbox{\sf zero}}
\def\SUCC{\hbox{S}}
\setlength{\oddsidemargin}{0in} % so, left margin is 1in
\setlength{\textwidth}{6.27in} % so, right margin is 1in
\setlength{\topmargin}{0in} % so, top margin is 1in
\setlength{\headheight}{0in}
\setlength{\headsep}{0in}
\setlength{\textheight}{9.19in} % so, foot margin is 1.5in
\setlength{\footskip}{.8in}

% Definition of \placetitle
% Want to do an alternative which takes arguments
% for the names, authors etc.

\newcommand{\lbr}{\lbrack\!\lbrack}
\newcommand{\rbr}{\rbrack\!\rbrack}
%\newcommand{\sem}[2] {\lbr #1 \rbr_{#2}}  % interpretation of the terms
\newcommand{\PAIR}[2] {{<}#1,#2{>}}  % interpretation of the terms
\newcommand{\add}{\mathsf{add}}
\newcommand{\app}{\mathsf{app}}
\newcommand{\APP}{\mathsf{APP}}
\newcommand{\BAPP}[2]{\mathsf{app}(#1,#2)}
\newcommand{\nat}{{N}}
\newcommand{\NNO}{\hbox{\sf N$_0$}}
\newcommand{\UU}{\hbox{\sf U}}
\newcommand{\VV}{\hbox{\sf V}}
\newcommand{\EXIT}{\mathsf{exit}}
\newcommand{\natrec}{\hbox{\sf{natrec}}}
\newcommand{\boolrec}{\hbox{\sf{boolrec}}}
\newcommand{\nil}{[]}
\newcommand{\cons}{\mathsf{cons}}
\newcommand{\lists}{\mathsf{list}}
\newcommand{\VEC}{\mathsf{vec}}
\newcommand{\reclist}{\mathsf{RecL}}
\newcommand{\vect}{\mathsf{vect}}
\newcommand{\brec}{\Phi}
\newcommand{\brecp}{\Psi}
\newcommand{\true}{\mathsf{true}}
\newcommand{\false}{\mathsf{false}}
\newcommand{\bool}{{N_2}}
\newcommand{\ifte}[3]{\mathsf{if}\ #1\ \mathsf{then}\ #2\ \mathsf{else}\ #3}
\newcommand{\nats}{\mathbb{N}}
\newcommand{\Con}{{\sf Con}}
\newcommand{\Typ}{{\sf Type}}
\newcommand{\Elem}{{\sf Elem}}
\newcommand{\Char}{{\sf Char}}
%\newcommand{\id}{{\sf id}}
\newcommand{\id}{{1}}
\newcommand{\pp}{{\sf p}}
\newcommand{\qq}{{\sf q}}
\newcommand{\comp}{{\sf comp}}
% Marc's macros
\newcommand{\op}[1]{#1^\mathit{op}}
\newcommand{\set}[1]{\{#1\}}
\newcommand{\es}{\emptyset}
\newcommand{\lto}{\longmapsto}
\newcommand{\rup}[1]{#1{\uparrow}}
\newcommand{\rdo}[1]{#1{\downarrow}}
\newcommand{\rupx}[1]{#1{\uparrow_{x}}}
\newcommand{\rdox}[1]{#1{\downarrow_{x}}}
\newcommand{\rupxy}[1]{#1{\uparrow_{x,y}}}
\newcommand{\rdoxy}[1]{#1{\downarrow_{x,y}}}
\newcommand{\rupyx}[1]{#1{\uparrow_{y,x}}}
\newcommand{\rdoyx}[1]{#1{\downarrow_{y,x}}}
\newcommand{\del}[1]{}
\newcommand{\ul}[1]{\underline{#1}}
%\newcommand{\bind}[2]{{\langle}#1.\, #2{\rangle}}
\newcommand{\bind}[2]{{\langle}#1{\rangle}#2}
\newcommand{\OO}{O}
\newcommand{\takeout}[1]{}
% end Marc's macros

\newcommand{\yoneda}{\mathbf{y}}


\begin{document}

\title{A model of type theory in cubical sets}

\author{
Marc Bezem\thanks{Department of Informatics, University of Bergen, {\tt bezem@ii.uib.no}}
\and
Thierry Coquand\thanks{Department of Computer Science and Engineering, Chalmers/University of Gothenburg, {\tt coquand@chalmers.se}}
\and
Simon Huber\thanks{Department of Computer Science and Engineering, Chalmers/University of Gothenburg, {\tt simonhu@chalmers.se}}
}
\date{\today}
\maketitle

%\rightfooter{}

{\em La th\'eorie singuli\`ere classique utilise des {\em simplexes}; dans la suite de ce chapitre,
nous aurons besoin d'une d\'efinition
\'equivalente, mais utilisant des {\em cubes}; il est en effet \'evident que ces derniers se pr\^etent mieux que les
simplexes \`a l'\'etude des produits directs, et, a fortiori, des espaces fibr\'es qui en sont la g\'en\'eralisation.}
(J.P. Serre, Th\`ese, Paris, 1951 \cite{Serre}).


\section*{Introduction}

 We present a model of dependent type theory in cubical sets.
 This provides a generalization to dependent types
of Takeuti/Gandy's interpretation of extensional in intensional type theory \cite{Gandy}, but also
a generalization of Bishop's notion of {\em set} \cite{Bishop}.
While not expressed internally in type theory, this model is expressed in a constructive metalogic.
It can be seen as a simplification and a constructive version of the
Kan simplicial set model of type theory \cite{Voevodsky}.
Our notion of cubical set is a variation of the presentation in \cite{Crans}.
We introduce the base category $\CC$ of names and substitutions.
A \emph{cubical set} is then a covariant presheaf on $\CC$.
We can then consider the presheaf model of type theory,
where a context $\Gamma$ is interpreted by a cubical set and a type over $\Gamma$ is interpreted by a fibration
over the interpretation of $\Gamma$. Like for the classical model based on simplicial sets,
we restrict our model by considering only Kan fibrations. A problem with a constructive approach to
Kan simplicial sets is that degeneracy is in general undecidable \cite{BC}.
Here we shall use a strengthening of the usual Kan filling condition. (One other possible attempt to solve this
problem is by marking the degenerated elements, but it is not clear yet in this approach how to treat exponentials
in a satisfactory way.) The strengthening of the Kan condition is natural given the reformulation of the notion of cubical sets
that we present in the first section. This reformulation opens connections with the study of nominal
sets \cite{pitts}; one connection is elaborated in \cite{Pitts}.

 The paper is structured as follows. First we introduce the category of names and substitutions.
Then we define cubical sets and the uniform Kan condition.
In Section~\ref{cubsetmodel} we show that cubical sets are a model for dependent types.
In the last section we show how identity types can be interpreted the cubical set model,
and describe the universe, and how to transform an equivalence in an equality of types.
Finally, we explain how to represent in our model spaces up to homotopy such as the sphere,
and the operation of propositional reflection, giving in particular a new computational interpretation
of the axiom of description \cite{Russell}.

\section{The category of names and substitutions}

 We start by fixing a countable discrete set of names or symbols, hereafter called the \emph{name space},
such that $0$ and $1$ are not names.
The objects of $\CC$ are finite decidable subsets of the namespace,
and we denote them by $I,J,K,\dots$.
A morphism $f:I\to J$ is a map $I \to J\cup \set{0,1}$ such that $f(i) = f(j)$ iff $i=j$ whenever
$f(i)$ and $f(j)$ are in $J$. Notice that $\set{0,1}$ is disjoint from $J$ since $J$ is a set of
names. We say that $i$ is in the \emph{domain} of $f$, or that $f(i)$ is \emph{defined},
if $f(i)$ is an element of $J$\footnote%
{In a previous attempt, we have been considering the category of finite sets
with maps $I \to J+2$ (i.e. the Kleisli category for the monad $I+2$).
This category appears on pages 47--48 in Pursuing Stacks
\cite{Grothendieck} as ``in a sense, the smallest test category''.}.
So the condition for $f$ being a morphism can be reformulated by saying that $f$ is injective on its domain.

Clearly, $1_I : I\to I$ defined by $1_I(i) = i$ for all $i\in I$ is a morphism.
If $f:I\to J$ and $g:J\to K$ are morphisms, we define the composition $g\circ f$ by
$(g\circ f)(i) = g(f(i))$ if $i$ is in the domain of $f$, and $(g\circ f)(i) = f(i)$ if $f(i)= 0,1$.
Clearly, $g\circ f: I\to K$ is a morphism.
We shall write $fg$ for the composition $g\circ f$, so first $f$ and then $g$.

Every $f:I\to J$ has a unique extension to a map $I \cup \set{0,1} \to J\cup \set{0,1}$
that is the identity on $\set{0,1}$, and this canonical extension respects composition.
Together with $I\mapsto I \cup \set{0,1}$ we get a functor.% $\CC\to{\sf Set}$.

It is not difficult to see that composition is associative and that $1_I f = f = f 1_J$.
Hence $\CC$ is a category. From now on, we may simply write $1$ instead of $1_I$.

 We think of $f:I\to J$ as a substitution with renaming,
 where the only values we can substitute are $0$ and $1$.
In particular we have for any $x$ in $I$ two substitutions $(x=b):I\to I-x$, for $b=0,1$,
defined by $y(x=b) = y$ if $y\neq x$ and $x(x=b) = b$. These are the \emph{face maps}.
Thus cubical sets have $2n$ face maps when $I$ has $n$ elements, that is, in dimension $n$
(where simplicial sets have $n+1$ face maps).

 We say that a map $f:I\to J$ is a {\em degeneracy map} iff all elements in $I$ are in
the domain of $f$. For instance, if $I\subseteq J$ the canonical inclusion $I \to J$ defines
a degeneracy map. If $x$ is not in $I$ the inclusion map $I \to I,x$ will be written as $(x)$.
We have two face maps $(x=0), (x=1):I,x\to I$ and we
have $(x)(x=0) = (x)(x=1) = 1_I$, which is one example of a \emph{cubical identity}.
There are many more cubical identities, often implicit in the notations. We also have
the following result (cf.\ simplicial sets): every morphism is, in a unique way, the composition of
face maps followed by degeneracy maps.

 If $f:I\to J$ is defined on $x$ then we write $f-x:I-x \to J-{f(x)}$ for the map defined by
$(f-x)(y) = f(y)$ if $y$ is in $I-x.$

 If $f:I\to J$ and $x$ is not in $I$ and $y$ is not in $J$ we can extend $f$ to a map
$(f,x=y):I,x\to J,y$ by sending $x$ to $y$.

\section{Cubical sets}

A \emph{cubical set} is a functor $\CC\to{\sf Set}$.
Let $X$ be a cubical set. Then we have sets $X(I)$ and set maps
$X(I)\to X(J),~u\longmapsto uf$ for any morphism $f:I\to J$,
such that $u1 = u$ and $u(fg) = (uf)g$.

\medskip

A cubical set $X$ is a presheaf on the category ${\cal C}^{opp}$. Any
finite set of directions $I$ represents by the Yoneda embedding
$\yoneda \colon {\cal C}^{opp} \to {\sf Set}^{\cal C}$ a cubical set
$\yoneda I$, which can be thought of as a formal representation of
$[0,1]^I$.  An element of $X(I)$ can then be seen as a formal
representation of a ``continuous'' map $[0,1]^I\to X$, and it is
natural to call an element of $X(I)$ an $I$-cube.

We think of $u$ in $X(I)$ as meaning that $u$ may depend on the names in $I$, and only on those names;
we think of $uf$ in $X(J)$ as the element we obtain by performing the substitution $f$ on $u$,
possibly combined with renaming and/or adding variables.
An element of $X()$ represents a point, an element $\omega$ of $X(i)$ a line connecting the points
$\omega(i=0)$ and $\omega(i=1)$ in $X()$. An element in $X(i,j)$ represents a square.
We then follow some notations similar to the ones in first-order logic by writing $u = u(x_1,\dots,x_n)$ when
$u$ is in $X(x_1,\dots,x_n)$. This is similar to saying that $u$ may depend at most on the names $x_1,\dots,x_n$.
In doing so we always implicitly assume that the names $x_1,\dots,x_n$ are pairwise distinct;
the order of the names in $X(x_1,\dots,x_n)$ does not matter. Applying a face map will now be expressed
by actually performing the substitution. For example, we have that $u(x=0)$ is in $X(y)$ whenever $u$ is in $X(x,y)$:
%\begin{figure}
\[
\xymatrix@R=1.1cm@C=.6cm{
u(0,0)\ar[rr]^{u(x,0)}\ar[dd]_{u(0,y)}&      & u(1,0)\ar[dd]^{u(1,y)}   &&&
u(0,0)\ar[rr]^{u(x,0)}\ar[dd]_{u(0,0)}&      & u(1,0)\ar[dd]^{u(1,0)}       \\
                                      &u(x,y)&                          &&&
                                      &u(x,0)&                              \\
u(0,1)\ar[rr]_{u(x,1)}                &      & u(1,1)                   &&&
u(0,0)\ar[rr]_{u(x,0)}                &      & u(1,0)                       \\
}
\]
%\caption{A square $u(x,y)$ and a degenerated square $u(x,0)$\label{squares}}
%\end{figure}

If $v$ is an $I-x$ cube of $X$ then we can consider $v(x)$ which is an
$I$-cube of $X$ (we recall that $(x):I-x\to I$ is the canonical
inclusion). The map $v\longmapsto v(x)$ is injective (we have
$v(x)(x=0) = v$) and it is sometimes natural to identify $v$ and
$v(x)$, thus considering $X(I-x)$ to be a subset of $X(I)$. An example
is the degenerated right square above.

 If $u$ is in $X(I)$ and $x$ is in $I$, there may exist a $v$ in $X(I-x)$ such that
$u = v(x) = v$.  Intuitively, this means that $x$ ``does not occur'' in $u$, or that $u$ is
``independent'' of $x$. One sometimes uses the notation $x\#u$ to express this relation. In general, this relation does
not need to be decidable.

 If $X$ is a cubical set and $a$ and $u$ are two points ($\es$-cube) of $X$ we can define a new cubical sets $\Id_X~a~u$
by taking an element in $(\Id_X~a~u)(I)$ to be an $I,x$-cube $\omega$ of $X$ where $x$ is
a fresh variable, such that $\omega(x=0) = a$ and $\omega(x=1) = u$. The name $x$ is ``bound''
in this operation so that another $I,x'$-cube $\omega'$ is equal to $\omega$
iff $\omega'(x'=x) = \omega$.
%More precisely, if $I=\set{x_1,\dots,x_n}$, then the morphisms
%$(x=x'): I,x \to I,x'$ and $(x'=x): I,x' \to I,x$ renaming only $x$ to $x'$ and vice versa are each other's inverse,
%and we can identify $\omega\in X(x_1,\dots,x_n,x)$ with $\omega'\in X(x_1,\dots,x_n,x')$
%whenever $\omega' = \omega (x=x')$, or, equivalently, $\omega = \omega' (x'=x)$.
%Due to the isomorphisms $x=x'$ and $x'=x$, the name of the bound variable does not matter (as long as it is not in $I$).
We introduce a new binding operation $\bind{x}{\omega}$ which defines this $I$-cube of $\Id_X~a~u$.
One way to make this notion precise is to assume a choice function
on the set of names which selects a fresh name for any finite subset and define $\bind{x}{\omega}$ to be $\omega (x=x_I)$ where
$x_I$ is the fresh name given by the choice function. (This is the solution suggested in \cite{Stoughton}.)

 Since cubical sets form a presheaf category, it is known how to give a semantics of type theory in term of cubical sets.
A context $\Gamma$ is interpreted by a cubical set, and a dependent type $\Gamma\vdash A$ is given by a fibration.
More explicitly,
we need to give for each $I$-cube $\alpha$ of $\Gamma$ a set $A\alpha$
with maps $A\alpha\to A\alpha f,~u\longmapsto uf$ satisfying $u1 = u$ and $u(fg) = (uf)g$.
A section $\Gamma\vdash a:A$ is defined by giving an element $a\alpha$ in $A\alpha$ for each
$I$-cube $\alpha$ of $\Gamma$ in such a way that $(a\alpha) f = a (\alpha f)$ for any substitution $f$.
If $\Gamma\vdash A$, we define the presheaf $\Gamma.A$ by taking as $I$-cubes of $\Gamma.A$
pairs $(\alpha,u)$ with $\alpha$ an $I$-cube of $\Gamma$ and $u$ in $A\alpha$. If $f:I\to J$ we define
$(\alpha,u)f = (\alpha f,u f)$. If $\Gamma\vdash A$ and $\Gamma.A\vdash B$ we define
the fibration $\Gamma\vdash \Pi\,A\,B$ as follows.
For each $I$-cube $\alpha$ of $\Gamma$, an element $w$ of $(\Pi\,A\,B)\alpha$ is a family
indexed by $f:I\to J$ such that $\app(wf,u)$ in $B(\alpha f,u)$ for all $u$ in $A\alpha f$.
Moreover, we should have $\app(wf,u)g = \app(wfg,ug)$ for all $g:J\to K$. The latter
allows us to define the family $wf$ by putting $(wf)g = w(fg)$, $w=w1$,
which completes the definition of the fibration $\Gamma\vdash \Pi\,A\,B$.
Substitutions $\sigma:\Delta\to\Gamma$ are interpreted as cubical set maps
(natural transformations), so that we have $(\sigma\beta)f = \sigma(\beta f)$ if $\beta$ is a $I$-cube of
$\Delta$.

\medskip

 The corresponding category with the same objects and morphisms $I \to J\cup\{0\}$ has been already
considered as the category of {\em partial injections}. It has been shown by Staton that the category of covariant
presheaves over this category is equivalent to the category of nominal sets with one restriction operation
(see \cite{pitts}, exercise 9.7). Using the same method, we can associate in a canonical way a nominal set
to any cubical set. A category equivalent to to the category of cubical sets is presented in \cite{Pitts}.

\section{The uniform Kan condition}

 Using these notations we can formulate the Kan condition (cf.\ \cite{Kan})
and our strengthening as follows. Let $X$ be a cubical set.
First we define the notion of an \emph{open box} in $X$, the equivalent of a \emph{horn} in a simplicial set.
Let $I$ be a finite set of names and let $J,x \subseteq I$.
The variable $x$ must not be in $J$ and will be the direction in which the box is open.
For every $y\in J$, the open box will have two faces, one with $y=0$ and one with $y=1$.
Let $O^+(J,x)$ consist of pairs $(x,0)$ and $(y,b)$ for $y\in J,~b=0,1$.
In the same way we define $O^-(J,x)$, but with $(x,1)$ instead of $(x,0)$.
The idea for both is that one face in the direction $x$ is missing.
We use $O(J,x)$ to denote either $O^+(J,x)$ or $O^-(J,x)$.
Given $O(J,x)$, we define an \emph{open box} as a vector $\vec u$ indexed by $O(J,x)$
such that the faces $u_{yb}$ in $X(I-y)$ are \emph{adjacent-compatible},
that is, for all $(y,b),(z,c)\in O(J,x)$ with $y\neq z$:
$$u_{yb}(z = c) = u_{zc}(y = b)$$
If $f:I\to K$ is defined on $J,x$ we write
$\vec u f$ for the vector $(\vec u f)_{yb} = u_{yb} (f-y)$.

 For $X$ to be a (constructive) Kan cubical set, we require to be given operations
$\rup{X} \vec u$ and $\rdo{X} \vec u$, both in $X(I)$, for every $J,x\subseteq I$.
Here $\vec u$ is an open box with $u_{x0}$ and $u_{x1}$ in $X(I-x)$ in the
respective cases $\rup{X} \vec u$ and $\rdo{X} \vec u$. (From now on we will always
tacitly assume that the open box $\vec u$ is of the right type with respect to $\rup{X},\rdo{X}$.)
The operations $\rup{X},\rdo{X}$ are to be thought of as a filling their
respective open boxes. Therefore we require for all $(y,b)\in O(J,x)$:
$$
(\rup{X} \vec u) (y=b) = u_{yb}~~~~~~~~(\rdo{X} \vec u) (y=b) = u_{yb}
$$
The new uniformity condition is: if $f:I\to K$ is defined on $J,x$, we require:
$$
(\rup{X} \vec u) f = \rup{X} (\vec{u}f)~~~~~~~~(\rdo{X} \vec u) f = \rdo{X} (\vec{u}f)
$$
We refer to the combined condition as the \emph{uniform Kan condition for cubical sets},
or the \emph{Kan condition} for short. %See Figure~\ref{fillings} for an example.

If we only consider the case where $I = J,x$, that is, no other
variables in $I$, and without the uniformity conditions, we get back
the usual notion of Kan cubical sets \cite[Section 4]{Kan} (adapted to
our notion of cubical ).  For a suggestive description of how to
define combinatorially $\pi_n(X,u)$ for each point $u$ of $X$ if $X$
satisfies the Kan property, see \cite{Williamson}.

 These new uniformity conditions solve the constructivity problem described in the note \cite{BC}.
(These uniformity conditions can always be satisfied if we assume decidablity of degeneracy, by a definition similar
as the one considered in \cite{BCH}.)

If $X$ is a Kan cubical set with operations $\rup{X},\rdo{X}$, we define new operations (see figure below)
$$
X^+\vec u = (\rup{X} \vec u)(x=1)~~~~~~~~X^-\vec u = (\rdo{X} \vec u)(x=0)$$
%\begin{figure}
%\def\emp{{()}}
\def\noe{\cdot}
\[
\xymatrix@R=1.1cm@C=.6cm{
\noe\ar@{.>}[rr]^{X^+(u_{x0}, u_{y0},u_{y1})}&      & \noe                                     &&&
\noe\ar[rr]^{u_{x1}}                                 &      & \noe      \\
                                      &\rup{X}(u_{x0}, u_{y0},u_{y1})&                          &&&
                                      &\rdo{X}(u_{x1}, u_{y0},u_{y1})&                              \\
\noe\ar[rr]_{u_{x0}}\ar[uu]^{u_{y0}}&      & \noe\ar[uu]_{u_{y1}}                                     &&&
\noe\ar@{.>}[rr]_{X^-(u_{x1}, u_{y0},u_{y1})}\ar[uu]^{u_{y0}}     &      & \noe\ar[uu]_{u_{y1}}                       \\
}
\]
%\caption{Two examples of filling open boxes\label{fillings}}
%\end{figure}

 We now define what is a Kan fibration over a cubical set.
Let $\Gamma$ be a cubical set (which does not need to satisfy the Kan condition)
and $\Gamma\vdash A$ a fibration over $\Gamma$.
For this fibration to be a \emph{Kan fibration}, we require to be given operations,
for any $J,x\subseteq I$,
$\rup{A\alpha} \vec u$  and $\rdo{A\alpha} \vec u$, both in $A\alpha$,
defined for every open box $\vec u$.
Here \emph{open box} means that $u_{yb}\in A\alpha(y=b)$ for all $(y,b)\in O(J,x)$,
and that these faces are adjacent-compatible.
$\rup{A\alpha}, \rdo{A\alpha}$ must satisfy the same Kan conditions as $ \rup{X},\rdo{X}$ above.
The usual Kan conditions are obtained by simply substituting $A\alpha$ for $X$.
Since $f:I\to K$ interacts with $\alpha$, we reformulate the uniformity conditions:
$$
(\rup{A\alpha} \vec u)f = \rup{A\alpha f} (\vec{u}f)~~~~~~~
(\rdo{A\alpha} \vec u)f = \rdo{A\alpha f} (\vec{u}f)
$$
If $\Gamma\vdash A$ is a Kan fibration with operations $\rup{A\alpha},\rdo{A\alpha}$, we define as before
$$
A\alpha^+\vec u = (\rup{A\alpha} \vec u)(x=1)~~~~~~~~
A\alpha^-\vec u = (\rdo{A\alpha} \vec u)(x=0)
$$

\section{Examples of cubical sets}\label{sec:examples}
In this section we elaborate the following examples of cubical sets:
the unit interval $I$;
the cubical nerve $N$ of the group $Z_2$ with two elements;
the exponential $N^I$.
A noticeable difference between simpicial sets and our cubical sets is that,
while $N$ is Kan, $N^I$ is not.
This is important motivation for the main result of the next section,
implying that $B^A$ is a Kan cubical set if both $A$ and $B$ are.

Recall the canonical extension of a map $f:J\to K$ in $\CC$
to a set map $J \cup \set{0,1} \to K\cup \set{0,1}$ that is the identity on $\set{0,1}$.
Together with  mapping objects $J$ of $\CC$ to $J \cup \set{0,1}$,
canonical extension actually forms a functor $\CC\to{\sf Set}$.
This covariant functor is called the \emph{unit interval}, denoted by $I$.
We explore: $I(\set{})=\set{0,1}$ ($I$ has two points);
$I(\set{x})=\set{0,1,x}$ ($I$ has three lines, only $x$ is non-degenerated);
$I(\set{x,y})=\set{0,1,x,y}$ ($I$ has four degenerated squares,
the first four squares in the display below); and so on.
The square $x$ varies in direction $x$,
but is constant in direction $y$, and hence degenerated.
Similarly for objects of higher dimension in $I$.
This completes the description of the unit interval as a cubical set.
\[
\def\olll{{1 \atop 0} ~ {1 \atop 1}}
\def\oool{{0 \atop 0} ~ {1 \atop 0}}
\def\oooo{{0 \atop 0} ~ {0 \atop 0}}
\def\llll{{1 \atop 1} ~ {1 \atop 1}}
\def\olol{{1 \atop 0} ~ {1 \atop 0}}
\def\ooll{{0 \atop 0} ~ {1 \atop 1}}
\def\Fxy{{0 \atop 0} ~ {0 \atop 0}}
\def\Txy{{1 \atop 1} ~ {1 \atop 1}}
\def\Iy{{1 \atop 0} ~ {1 \atop 0}}
\def\Ix{{0 \atop 0} ~ {1 \atop 1}}
I(\set{x,y}):\Fxy\quad\Txy\quad\Ix\quad\Iy\quad\quad
N(0\leq1)(\set{x,y}):\oooo\quad\ooll\quad\olol\quad\llll\quad\olll\quad\olll\]

Recall that a morphism $f:J\to K$ in $\CC$ is a function $f:J\to K\cup\set{0,1}$ such that
for every $y\in K$ there exists at most one $x\in J$ with $f(x)=y$.
Hence every morphism $f:J\to K$ defines a function
$\set{0,1}^K  \to \set{0,1}^J$ through precomposition with $f$.
We can view $\set{0,1}^J$ as a product of posets $0\leq1$,
and hence as a category with unique morphisms.
Then every morphism $f:J\to K$ defines a functor
$\set{0,1}^K  \to \set{0,1}^J$, as the precomposition preserves the order.
We denote this functor also by $f$.

\newcommand{\DD}{{\cal D}}
\newcommand{\ND}{N\DD}
Given a small category $\DD$, we define its \emph{cubical nerve}
$\ND$ as follows. The sets $\ND(J)$ are functors $\set{0,1}^J \to\DD$.
For every morphism $f:J\to K$, its function $\ND(J)\to\ND(K)$
is defined by precomposition with the functor $f$.
Note that the unit interval is not the cubical nerve of the poset $0\leq1$:
they have similar sets of points and lines, but $N(0\leq1)$ has two
more squares, both non-degenerated in two directions,
see the last six squares in the display above.
An element of $\ND(J)$ can be viewed as a (hyper)cube with
the edges labelled by morphisms of $\DD$ such that:
(i) all begin- and endpoints match, and
(ii) all paths commute (by uniqueness of morphisms in the poset).
Point (ii) is equivalent to saying that all squares commute.
This completes the description of the cubical nerve of a category.

Consider the group of two elements as a category (groupoid)
with one object $\star$ and two morphisms $0,1:\star\to\star$.
Let $N$ be the nerve of this groupoid:
$N$ has one point and two lines, again denoted by $\star$ and $0,1$.
Note that $\star(x) = 0$ and $1\circ 1 = 1+1=0$.
The squares of $N$ are listed as follows, where we only show the lines:
\[
\def\ol{{1 \atop 0}}
\def\oo{{0 \atop 0}}
\def\lo{{0 \atop 1}}
\def\ll{{1 \atop 1}}
N(\set{x,y}){:}~
0\oo0 \quad 1\oo1 \quad 0\ll0 \quad 1\lo0 \quad 1\ol0 \quad 0\ol1 \quad 0\lo1 \quad 1\ll1
\quad\mbox{For later:}~0{\ell_x \atop \ell_x}0 \quad 0{\ell_1 \atop \ell_0}0
\quad \ell_0{0 \atop 0}\ell_1 \quad \ell_y{0 \atop 0}\ell_y\]
Being the nerve of a groupoid, $N$ is Kan (see the next section).

We now show that $N^I$ is not Kan. By the Yoneda Lemma we have
$N^I(J)\cong (Hom^{\CC}(J,\_)\to (I\to N))$, the latter denoting a set
of natural translations.  Defining $p\in N^I(J)$ means defining maps
$pf:I(K)\to N(K)$ for every $f:J\to K$, such that $(pf)g=p(fg)$ for
every $g:K\to L$. We explore the points of $N^I$ and define $p\in
N^I(\emptyset)$ by, first, $p():I(\emptyset)\to
N(\emptyset):\_\mapsto\star$.  Then, $p(x):I(\set{x})\to N(\set{x}):
0,1 \mapsto\star(x)=0$ is forced by naturality, but for $p(x)(x)$
there is a choice. If we choose $0$, we must make the same choice for
all names $x$ in the name space.  The choice $1$ for all names $x$ in
the name space would give the only other point. In higher dimensions
all arguments are degenerated, determining the function values, and
naturality is compatible with each of the two choices above.

Next we explore lines from $p$ to $p$ in $N^I$,
say in direction $i$,
and define $\ell:p\to p$ in $N^I(\set{i})$ by
$\ell(i=b) = p()$ and $\ell(i=b)(x) = p(x)$ (forced, $b=0,1$).
For $\ell(i=x): I(\set{x})\to N(\set{x})$ there is a choice. For the moment
we put $\ell(i=x)(c) = \ell_c$ for all $c\in I(\set{x})$. Note that we must make
the same choices $\ell_0,\ell_1,\ell_x$ for all names $x$ in the name space.
On the next level, $\ell(i=b)(x)(y) = p(x)(y)$ is forced, as
are $\ell(i=x)(y),\ell(i=y)(x): I(\set{x,y})\to N(\set{x,y})$.
Applying these to both $x$ and $y$ in $I(\set{x,y})$ results
in the four squares displayed above (NB: $\ell_x=\ell_y$).
Naturality now limits the choice on the lower level,
since the squares have to commute: $\ell_0 = \ell_1$.
In higher dimensions all values are determined by naturally,
and naturality is compatible
with each of the four possible choices (recall that
objects in $I$ can be non-degenerated in at most one direction).
This yields in total four lines from $p$ to $p$ in $N^I$.

In order to show that $N^I$ is not Kan, consider lines $p,\ell: p\to p$,
where $p$ is degenerated ($p_0=p_x=p_1=0$)
and $\ell$ is defined by $\ell_0=\ell_1=0,~\ell_x=1$.
Consider an open box as in the picture below, left:
\[
\xymatrix@R=.6cm@C=.6cm{
p\ar@{.>}[rr]&      & p   &  0\ar[rr]^{0}&      & 0  &  \star\ar[rr]^{\ell'_0}&      & \star
&  1\ar[rr]^{1}&      & 1 &  \star\ar[rr]^{\ell'_1}&      & \star \\
                    &&    &        &&    &    &&       &        &&    &    &&       \\
p\ar[rr]_{p}\ar[uu]^{p}& & p\ar[uu]_{\ell} &
0\ar[rr]_{0}\ar[uu]^{0}& & 0\ar[uu]_{0}    &
\star\ar[rr]_{0}\ar[uu]^{0}& & \star\ar[uu]_{0}  &
0\ar[rr]_{0}\ar[uu]^{y}& & 0\ar[uu]_{y}    &
\star\ar[rr]_{0}\ar[uu]^{0}& & \star\ar[uu]_{1}     \\
}
\]
By contradiction, assume we can fill the box.
Call the closing (dotted) line above $\ell'$.
Applying the first square to the second results in the third square,
yielding $\ell'_0 = 0$.
Applying the first square to the fourth results in the last square,
yielding $\ell'_1 = 1$.
This conflicts with $\ell'_0 = \ell'_1$ for any line $p\to p$.

\subsection{The nerve of a groupoid is Kan}\label{sec:groupoidnerve}
Let $G$ be a groupoid, and $N$ its cubical nerve.
We prove that $N$ is Kan. Take $I=x,J,z$ in $\CC$,
with $J=y_1,\ldots,y_k$ $(k\geq 0)$. Taking one variable $z$ instead of
$z_1,\ldots,z_n$ simplifies the presentation, but is otherwise inessential.

Let $\vec u$ be an open box indexed by $O(J,x)$, that is,
adjacent-compatible faces $u_{x0}\in N(I-x)$ and $u_{yb}$ in $N(I-y)$.
We have to define $u\in N(I)$ with faces as given by the open box.
For this we define closing faces $u_{x1}$,$u_{z0}$,$u_{z1}$, such that they
are adjacent-compatible with the open box, and show that all squares
commute. This will define $u$ in a unique way.
Thereafter we shall verify the uniformity condition.

If $J=\emptyset~(k=0)$,  the open `box' is a degenerated line $u_{x0}$.
We close by taking $u_{x1}=u_{z0}=u_{z1}=u_{x0}$, and $u$ is the
doubly degenerated square. If $J\neq\emptyset~(k>0)$,
we observe that all the points of $u$ are already given by the open
box, so that we can limit attention to the edges.
Moreover, if $J$ consists of more than one variable, all edges are also
already present in the open box, which makes the definition of the closing faces
particularly simple. This can be seen as follows.
For $b=0,1$, the faces $u_{y_1b}$ contain all edges in which $y_1 = b$,
and the faces $u_{y_2b}$ contain all edges in which $y_2 = b$.
In particular, the two faces $u_{y_2b}$ contain all edges in direction $y_1$.
Hence, the four faces $u_{y_1b}$, $u_{y_2b}$ together contain all edges.
For this reason we can take $I=x,y,z,~J=y$ and define edges in $u_{x1}$ in direction $y$.
This situation is depicted below, left, with the new edges as defined right.
The new edges make essential use of the inverses in the groupoid
and are uniquely defined.
\[
\xymatrix@R=.6cm@C=.6cm{
\ar@{.>}[rrrr]            &&&&         &
&&&&&         &
\ar@{.>}[rrrr]^{g_0^{-1}\cdot g_1\cdot g_2}           &&&&         \\
&\ar[lu]\ar[rr]        &&  \ar[ru]&   &
&&&&&   &
&\ar[lu]_{g_0} \ar[rr]_{g_1}         &&  \ar[ru]^{g_2} &   \\
&&u_{x0}&&                               &
&&&&&                               &
&&~~~&&                               \\
&\ar[ld]\ar[rr]\ar[uu]&&\ar[rd]\ar[uu]& &
&&\ar[ld]_x\ar[rr]^{y}\ar[uu]^z&&& &
&\ar[ld]^{g_3}\ar[rr]^{g_4}&&\ar[rd]_{g_5} & \\
\ar[uuuu]\ar@{.>}[rrrr]&&&&\ar[uuuu] &
&&&&&  &
\ar[uuuu]\ar@{.>}[rrrr]_{g_3^{-1}\cdot g_4\cdot g_5}&&&&\ar[uuuu]\\
}
\]
The new squares $u_{zb}$ commute as per construction.
Moreover, the new square $u_{x1}$ commutes since it can
be projected down to the commuting square $u_{x0}$
along edges that are invertible.
A similar argument can be used if $J$ contains more variables.
This completes the construction of $u\in N(I)$.

Uniformity will be shown to be a consequence of the uniqueness of
$u$ constructed above, and the following easy lemma.
This lemma can be useful other places as well.
\begin{lemma}  For all morphisms $f:I\to K$ in $\CC$ defined on $x$ we have
(i) $(x=b)(f-x) = f (f(x)=b)$ and (ii) $(x)f = (f-x)(f(x))$.
\end{lemma}
Now let $u=\rup{N}(u_{x0},u_{y0},u_{y1})$ and
$u'=\rup{N}(u_{x0}(f-x),u_{y0}(f-y),u_{y1}(f-y))$.
We have to show $u'=uf$.
By the lemma we have $u_{x0}(f-x) = uf(f(x)=0)$ and $u_{yb}(f-y) = uf(f(y)=b)$.
This means that $u'$ and $uf$ agree on the open box defining $u'$,
so they are equal by uniqueness.
Again, a similar argument can be used if $J$ contains more variables.
This completes the proof that the cubical nerve of a groupoid is Kan.

\section{The cubical set model}\label{cubsetmodel}

We call MLTT the version of Type Theory with rules as presented in Figure~\ref{MLTT}.
This is inspired by the presentation in \cite{Curien} (see also \cite{Dybjer}).
We check that all the rules of MLTT are valid.

\subsection{Dependent product}

\begin{theorem}
If both $\Gamma\vdash A$ and $\Gamma.A\vdash B$ satisfy the Kan condition then so does $\Gamma\vdash\Pi~A~B$.
\end{theorem}

\begin{proof}
Let $I$ be given, $\alpha$ an $I$-cube in $\Gamma$, and let $x\in I$.
We present the argument in the case $J=\es$, the general case is not essentially more difficult.
Also, as the cases $\rup{},\rdo{}$ are perfectly symmetric, we restrict attention to $\rup{}$.
We have to define $w=\rup{(\Pi~A~B)\alpha} w_0$ for all $w_0 \in (\Pi~A~B)\alpha(x=0)$.
Informally, $w_0$ forms an open box (as $J=\es$), which has to be closed and filled.
In the presheaf semantics, we should define $wf\in(\Pi~A~B)\alpha f$ for all $f:I\to J$,
such that $w=w1$, $w(fg) = (wf)g$ for any $g:J\to K$.
We proceed by first defining $\ul{w}(\alpha,w_0)$, which is
the applicative behaviour of $w$, for all $\alpha$ and $w_0$.
We then define $wf=\ul{w}(\alpha f,w_0(f-x))$ for all $f:I\to J$.
The definition of $\ul{w}$ will be such that we can calculate:
\[
(wf)g = (\ul{w}(\alpha f,w_0(f-x)))g = \ul{w}(\alpha fg,w_0(f-x)(g-f(x))) =  \ul{w}(\alpha fg,w_0(fg-x)) = w(fg)
\]

Instead of  $\ul{w}(\alpha,w_0)$ we write $\rup{(\Pi~A~B)\alpha} w_0$,
trading readability for a slight abuse of notation.
The applicative behavior should satisfy the following diagram for all $u\in A\alpha$:
\[
\xymatrix@R=.6cm@C=.6cm{
w_1                                         &:&   u_1\del{\in A\alpha(x=1)}             &\lto &   \app(w_1,u_1)\del{\in B(\alpha(x=1),u_1)}  \\
\rup{(\Pi~A~B)\alpha}~w_0 &:&    ~~~u\del{\in A\alpha}                  &\lto &    \app(\rup{(\Pi~A~B)\alpha}~w_0,u)\del{\in B(\alpha,u)}   \\
w_0\ar@{.>}[uu]                   &:&    u_0\del{\in A\alpha(x=0)}\ar[uu] &\lto &    \app(w_0,u_0)\del{\in B(\alpha(x=0),u_0)}\ar@{.>}[uu]   \\
}
\]
It is convenient to first define $w_1$ in $(\Pi~A~B)\alpha(x=1)$, closing the box.
For this we define $\app(w_1,u_1)$ for any $u_1\in A\alpha(x=1)$.
Intuitively, using that $\Gamma\vdash A$ and $\Gamma.A\vdash B$ satisfy the Kan
condition for $J=\es$, we map $u_1$ down to $A\alpha^- u_1$,
apply $w_0$ and map the result up:
$$
\app(w_1,u_1) = B(\alpha,\rdo{A\alpha}u_1)^+\app(w_0,A\alpha^- u_1)
$$
With $\lambda=\rup{B(\alpha,\rdo{A\alpha}u_1)} \app(w_0,A\alpha^- u_1)$, the situation can be depicted as follows:
\[
\xymatrix@R=.6cm@C=.6cm{
 u_1            && &   \app(w_1,u_1)  \\
 ~~~u               && &    \\
u_0\ar[uu]      &A\alpha^- u_1\ar[uul]_{\rdo{A\alpha}u_1}&\lto &    \app(w_0,A\alpha^- u_1)\ar[uu]_\lambda  \\
}
\]
Note that $A\alpha^- u_1 \in A\alpha(x=0)$ is not necessarily equal to $u_0$
which means that filling the box requires an additional argument.
It is here we need the extra uniformity condition,
using that $\Gamma\vdash A$ and
$\Gamma.A\vdash B$ satisfy the Kan condition for $J=\es$ and $|J|=1$.

We explain how to define for every $u\in A\alpha$
$$
\app(\rup{(\Pi~A~B)\alpha} w_0,u)
$$
We write $u_0 = u(x=0)$ and $u_1 = u(x=1)$.
Let $w_1$ be as defined above.
The line $\app((\Pi~A~B)\alpha\uparrow w_0,u)$ must
connect $\app(w_0,u_0)$ and $\app(w_1,u_1)$.
Even though $u_0$ and $A\alpha^- u_1$ may be different, they can
be connected in a new direction, for which we choose a fresh name $y$ not occurring in $I$.

Until now we have used $\rup{},\rdo{}$ without making explicit their dependence
on $I,J,x$. With the extended $I,y$ we use $\rupxy{}$ in the case $(I,y),\set{x},y$,
and $\rupyx{}$ in the case $(I,y),\set{y},x$. Similarly for $\rdo{},{}^+,{}^-$.

Using the uniform Kan property of $\Gamma\vdash A$ for $J=\set{y}$
we define $v$ in $A\alpha(y)$ by
$$
v = \rdoxy{A\alpha(y)}(u_1,u,\rdo{A\alpha} u_1)
$$
The line $v(x=0)$
connects then $u_0$ and $A\alpha^-u_1$ in the direction $y$ since we have
$$
v(x=0,y=0) = u_0~~~~~~~~~~~v(x=0,y=1) = A\alpha^- u_1
$$

Using the uniform Kan property of $\Gamma.A\vdash B$ for $J=\set{x}$ we define
$$
\app(\rup{(\Pi\,A\,B)\alpha} w_0,u) =
{B(\alpha(y),v)}^-_{y,x}(\lambda,\app(w_0,v(x=0)),\app(w_1,u_1))
$$
with $\lambda$ as above,
so that $\lambda(x=0) = \app(w_0,A\alpha^- u_1)$ and $\lambda(x=1) = \app(w_1,u_1)$.

The situation can now be depicted as follows:
\[
\xymatrix@R=.6cm@C=.6cm{
&u_1&                                                                          &&  &\app(w_1,u_1)&\\
u_1\ar[ur]^{u_1}&&                                               &&  \app(w_1,u_1)\ar[ur]^{ \app(w_1,u_1)~~~}&&\\
&v&                                                                          &\lto& &   &\\
&&A\alpha^- u_1\ar[uuul]_{\rdo{A\alpha}u_1}     &&  &&\app(w_0,A\alpha^- u_1)\ar[uuul]_{\lambda} \\
&u_0\ar[uuul]^u \ar@{.>}[ur]_{v(x=0)}&                 &&  &\app(w_0,u_0)\ar@{.>}[uuul]^{\app(\rup{(\Pi A B)\alpha} w_0,u)} \ar[ur]_{~~~~\app(w_0,v(x=0))}&\\
}
\]

Now, if $f:I\to K$ is defined on $x$ we define
$(\rup{(\Pi~A~B)\alpha} w_0) f = \rup{(\Pi~A~B)(\alpha f)} (w_0 (f-x))$.
If $f$ is not defined on $x$, we can write $f=(x=b)f'$ for some $f':I-x \to K$.
Then we can simply define $(\rup{(\Pi~A~B)\alpha} w_0) f = w_b f'$ for $b=0,1$.
%Uniformity to be verified, plus some other pending proof obligations.
%How much do we include in the paper?
\end{proof}

\subsection{Sum type}

 The interpretation of $\Gamma\vdash\Sigma~A~B$ given $\Gamma\vdash A$ and $\Gamma.A\vdash B$
is simpler than the interpretation of the dependent product. If $\rho$ is an $I$-cube
of $\Gamma$ we define $(\Sigma~A~B)\rho$ to be the set of pairs $u,v$ where $u$ is
an element of $A\rho$ and $v$ an element of $B(\rho,u)$. We define $(u,v)f = uf,vf$
if $f:I\to K$ is a substitution. Finally we define for $\vec p$
with $p_{yb}=(u_{yb},v_{yb})$ for any $(y,b)\in O(J,x)$:

$$(\rup{\Sigma~A~B)\rho}~\vec{p}~=~ \alpha, \rup{B(\rho,\alpha)}~\vec{v},~~
\text{where}~\alpha = \rup{A\rho}~ \vec u.$$
This operation is uniform: if $f:I\to K$ defined on $x$ we have:
$$(\rup{(\Sigma~A~B)\rho}~\vec{p})f~=~\rup{(\Sigma~A~B)\rho f}~(\vec{p}f)$$

\medskip

\section{Extensions}

\subsection{Identity type}

 We describe the interpretation of $\Gamma\vdash\Id_A~a~b$ given $\Gamma\vdash A$ and $\Gamma\vdash a:A$
and $\Gamma\vdash b:A$. Given an $I$-cube $\alpha$ in $\Gamma$ we define $(\Id_A~a~b)\alpha$ to be the set of
elements $\bind{x}{\omega}$ in $A\alpha(x)$ where $x$ is a fresh variable not in $I$, such that $\omega(x=0) = a\alpha$
and $\omega(x=1) = b\alpha$. The latter situation is conveniently described by $\omega: a\rho\to_x b\rho$.
We recall that $(x)$ denotes the canonical injection $I \to I,x$. The element $\bind{x}{\omega}$ is the equivalence class of
$I,x$-cubes of $A\alpha(x)$, $x$ not in $I$, where $\omega$ is identified with
$\omega(x=x')$ if $x'$ is not in $I$. This operation $\bind{x}{\omega}$ binds the name $x$.
(One could define $\bind{x}{\omega}$ to be $\omega(x = x_I)$ where $x_I$ is a name not in $I$ obtained
by a choice function.) If $f$ is a substitution $I \to K$ we choose a variable
$y$ not in $K$, extend $f$ to $(f,x=y): I,x \to K,y$ and define $(\bind{x}{\omega}) f$ to be
$\bind{y}{\omega (f,x=y)}$, preserving equivalence.

\begin{theorem}
If $\Gamma\vdash A$  satisfies the Kan condition then so does $\Gamma\vdash\Id_A~a~b$ whenever we have
$\Gamma\vdash a:A$ and $\Gamma\vdash b:A$.
\end{theorem}

\begin{proof}
Let $\alpha$ be an $I$-cube of $\Gamma$ and $J,x\subseteq I$. After a suitable renaming, we can conveniently
denote an open box for $(\Id_A~a~b)\alpha$ by a vector $\bind{y}{\vec{\omega}}$ with components
$\bind{y}{\omega_{zc}}\in(\Id_A~a~b)\alpha(z=c)$, for all $(z,c)\in O(J,x)$.

We define, with $a\alpha,b\alpha$ the faces in the direction $y$, omitting subscripts $J$,
$$
\rupx{(\Id_A~a~b)\alpha}\bind{y}{\vec{\omega}} =
\bind{y}{(\rupxy{A\alpha(y)} (\vec{\omega},a\alpha,b\alpha))}
%(\Id_A~a~b)\alpha\uparrow _{x,x_1,\dots} (\bind{y}\omega_0,\bind{y}\omega_0^1,\bind{y}\omega_1^1,\dots) =
%\bind{y}A\alpha(y)\uparrow _{x,y,x_1,\dots} (\omega_0,a\alpha,b\alpha,\omega_0^1,\omega_1^1,\dots)
$$
which shows that $\Gamma\vdash \Id_A~a~b$ satisfies the Kan condition for $J,x$ if $A$ satisfies the Kan
condition for $J,y,x$. The situation in case $J=\es$ is depicted below. For $f:I\to K$ defined on $x$ we avoid
caption by assuming $y$ not in $K$ and put:
$$
(\rupx{(\Id_A~a~b)\alpha} \bind{y}{\vec{\omega}})f =
\rupxy{(\Id_A~a~b)\alpha f} (\bind{y}{\vec{\omega}f}).
$$
\end{proof}
\[
\xymatrix@R=1.1cm@C=.6cm{
a\alpha\ar@{.>}[rr]^{A\alpha(y)_{x,y}^+(w_0,a\alpha,b\alpha)}&      & b\alpha                                     &&&
\bind{y}{\omega_1}                                &          \\
                                      &\rupxy{A\alpha(y)}(w_0,a\alpha,b\alpha)&                          &&&
                                      \rupx{(\Id_A~a~b)\alpha} \bind{y}{\omega_0}&                           \\
a\alpha\ar[rr]_{w_0}\ar[uu]^{a\alpha}&      & b\alpha\ar[uu]_{b\alpha}                                     &&&
\bind{y}{\omega_0}\ar@{.>}[uu]     &                             \\
}
\]

 We give the interpretation of $\Gamma\vdash \Ref~a:\Id_A~a~a$ given $\Gamma\vdash a:A$. For any set of
directions $I$, and any $I$-cube $\rho$, we have to give a line $a\rho\to a\rho$. For this, we
choose a direction $x$ not in $I$ and we define $(\Ref~a)\rho = \bind{x}{a\rho (x)}$,
which can also simply be written $(\Ref~a)\rho = \bind{x}{a\rho}$.

 Next we show that we can interpret an elimination operator for the identity type.
 Suppose $\Gamma\vdash a:A,~\Gamma\vdash b:A,~\Gamma\vdash u:\Id_A~a~b$ and
$\Gamma.A\vdash P$ and $\Gamma\vdash v:P[a]$. We will define an operator
$$
\Gamma\vdash \JJ(u,v):P[b].
$$
Let $\rho$ be some $I$-cube of $\Gamma$. Then $u\rho$ is of the form $\bind{x}{\omega}$
for some path $\omega:a\rho\to _x b\rho$, $x$ not in $I$, $\omega\in A\rho(x)$.
The $I,x$-cube $\rho (x),\omega$ in $\Gamma.A$
is then a path $[a]\rho\to _x [b]\rho$ and we define (see the picture below)
$$
\JJ(u,v)\rho = P(\rho (x),\omega)^+ v\rho~~~~{\rm where~~~~~}\bind{x}{\omega} = u\rho
$$
The condition $(\JJ(u,v)\rho)f = \JJ(u,v)(\rho f)$ follows from the uniformity condition on
the Kan filling operations.

\[
\xymatrix@R=.6cm@C=.6cm{
b\rho                 &&[b]\rho                    &   \JJ(u,v)\rho                                 \\
\omega ~~~     &&\rho(x),\omega~~  &\rup{P(\rho(x),\omega)} v\rho \\
a\rho\ar[uu]      &&[a]\rho\ar[uu]       &   v\rho\ar@{.>}[uu]                        \\
}
\]
We have that $\rup{P(\rho(x),\omega)} v\rho$ is a line
connecting $v\rho$ and $\JJ(u,v)\rho$.
In particular for $u=\Ref~a$, this gives an interpretation of an operator
$$
\Gamma\vdash \HH(v):\Id_{P[a]}~v~\JJ(\Ref~a,v)
$$
by taking $\HH(v)\rho = \bind{x}{\rup{P(\rho (x),a\rho)} v\rho}$.
The computation rule for identity is thus only validated  by a path to  $v$
via $\HH(v)$\footnote{The validity of the computation rule for identity corresponds to considering only fibrations
that are {\em regular} in the sense of Hurewicz \cite{hurewicz}.}.
N.A.~Danielsson has checked formally in Agda that these properties are
enough to develop all basic propositions of univalent mathematics.

 We finally show that, given $\Gamma\vdash a:A$,  the type $\Gamma \vdash T = \Sigma~A~(\Id_{A\pp}~a\pp~\qq)$ is
contractible. For this we have to find a center of $T$ and a path to this center for any element of $T$.
That is, we have to find two sections $\Gamma\vdash t:T$ and $\Gamma.T\vdash u:\Id_{T\pp}~t\pp~\qq$.
We define $t = (a,\Ref~a)$. Let $\rho$ be some $I$-cube of $\Gamma$
and let $(v,\bind{x}{\alpha})$ be some element of $T\rho$. So $v$ is
an element of $A\rho$ and $\alpha$ is a line connecting $a\rho$ and $v$ in some direction $x$ not in $I$. We
introduce a direction $y$ not in $I,x$ and define, stipulating degeneracies:
$$
u(\rho,(v,\bind{x}{\alpha})) = \bind{y}{(A\rho(y)^+ _{x,y} (a\rho(y),a\rho(x),\alpha),\bind{x}{\rupxy{A\rho(y)} (a\rho(y),a\rho(x),\alpha)})}
$$
The fact that the filling operations commute with substitution ensure that this defines
a section  $\Gamma.T\vdash u:\Id_{T\pp}~t\pp~\qq$.

\medskip

This interpretation of identity satisfies function extensionality.

\begin{figure}[t]
\caption{Rules of MLTT\label{MLTT}}
\centering
$$
\frac{\Gamma\vdash}{1:\Gamma\to \Gamma}~~~~~~~~
\frac{\sigma:\Delta\to\Gamma~~~~\delta:\Theta\to\Delta}
     {\sigma\delta:\Theta\to\Gamma}
$$
$$
\frac{\Gamma\vdash A~~~~~~\sigma:\Delta\to\Gamma}{\Delta\vdash A\sigma}~~~~~
\frac{\Gamma\vdash t:A~~~~~~\sigma:\Delta\to\Gamma}{\Delta\vdash t\sigma:A\sigma}~~~~~
$$
$$
\frac{}{()\vdash}~~~~~~
\frac{\Gamma\vdash~~~~~\Gamma\vdash A}{\Gamma.A\vdash}~~~~~~~~~~
\frac{\Gamma\vdash A}{\pp:\Gamma.A\to\Gamma}~~~~~~~~~
\frac{\Gamma\vdash A}{\Gamma.A\vdash \qq:A\pp}
$$
$$
\frac{\sigma:\Delta\to\Gamma~~~~\Gamma\vdash A~~~~~\Delta\vdash u:A\sigma}
     {(\sigma,u):\Delta\to\Gamma.A}~~~~~~~~
$$
$$
\frac{\Gamma.A\vdash B}{\Gamma\vdash\Pi~A~B}~~~~~~~~
\frac{\Gamma.A\vdash B~~~~\Gamma.A\vdash b:B}
     {\Gamma\vdash\lambda b:\Pi~A~B}
%\frac{\Gamma.A\vdash b:B}{\Gamma\vdash\lambda b:\Pi~A~B}
$$
$$
\frac{\Gamma.A\vdash B}{\Gamma\vdash\Sigma~A~B}~~~~~~~~
\frac{\Gamma.A\vdash B~~~~\Gamma\vdash u:A~~~~\Gamma\vdash v:B[u]}
     {\Gamma\vdash (u,v):\Sigma~A~B}
$$
$$
\frac{\Gamma\vdash w:\Pi~A~B~~~~~~\Gamma\vdash u:A}
     {\Gamma\vdash \app(w,u):B[u]}
$$
$$
\frac{\Gamma\vdash w:\Sigma~A~B}{\Gamma\vdash \pp w : A}~~~~~~~
\frac{\Gamma\vdash w:\Sigma~A~B}{\Gamma\vdash \qq w : B[\pp w]}
$$


\medskip

$$
{1\sigma = \sigma 1 = \sigma}~~~~~~{(\sigma\delta)\nu = \sigma(\delta\nu)}~~~~~~~~[u] = (1,u)
$$
$$
{A 1 = A \quad (A \sigma) \delta = A (\sigma \delta) \qquad  u 1 = u \quad
(u \sigma) \delta = u (\sigma \delta)}
$$
$$
{(\sigma,u)\delta = (\sigma\delta,u\delta)}~~~~~~{\pp (\sigma,u) = \sigma}~~~~~~~~~~~~{\qq(\sigma,u) = u}
$$
$$
     {\app(w,u)\delta= \app(w\delta,u\delta)}~~~~~~~~~~~~~
     {\app(\lambda b,u)= b[u]}
$$
$$
(\Pi~A~B)\sigma = \Pi~(A\sigma)~(B(\sigma\pp,\qq))~~~~~~~~~~
(\Sigma~A~B)\sigma = \Sigma~(A\sigma)~(B(\sigma\pp,\qq))
$$
\end{figure}

\subsection{Description of a universe}

% We describe the cubical structure on the universe. The points are
% cubical sets.  The lines $A\to B$ are obtained by having a
% ``heterogeneous'' notion of lines, cubes, $\dots$ $a\to b$
% where $a$ is an $I$-cube of $A$ and $b$ an $I$-cube of $B$, notion
% which is closed by renaming and face operations. To get the Kan model,
% we impose the uniform Kan property for this notion of cubes as well.

% It is then possible purely combinatorially to define compositions of
% lines, squares, $\dots$ and check that this cubical set satisfies the
% uniform Kan condition.

We now describe the interpretation of $U$ as a universe of Kan cubical
sets. Recall that the Yoneda embedding is denoted by $\yoneda$.  An
element $A$ of $U(I)$ is a Kan fibration $\yoneda I \vdash A$ such
that for each $f \colon I \to J$ the set $A_f$ is small (we use
subscripts to keep the notation separate from the restrictions).
Given such a $\yoneda I \vdash A$ and $f \colon I \to J$ the
restriction $A f$ of $A$ by $f$ is defined to be $\yoneda J \vdash A
(\yoneda f)$, where $\yoneda f \colon \yoneda J \to \yoneda I$ is the
substitution induced by $f$; thus $(A f)_g = A_{f g}$.  This defines
$U$ as a cubical set.

% More concretely, such an element $A$ is given by any family of
% (small) sets $A_f$ for $f \colon I \to J$, with maps $A_f \to
% A_{fg}$, $a \mapsto a g$ if $g \colon J \to K$ such that $(a g) h =
% a (g h)$; or, equivalently, $A$ is a functor from the coslice $\CC
% \backslash I$ to the category of small sets.  Moreover, we require
% operations similar to the uniform Kan conditions: For any $f \colon
% I \to K$ and $J, x \subseteq K$ with $x \notin J$, and $\vec a$ an
% open box in $A_f$, i.e., $a_{yb} \in A_{f (y=b)}$ for each $(y,b)
% \in \OO^+ (J,x)$ such that
% \[
% a_{yb} (z=c) = a_{zc} (y=b) \quad \text{for }(y,b),(z,c) \in \OO^+
% (J,x), y \neq z;
% \]
% we require $\rup{A_f} \vec a \in A_f$ such that for $(y,b) \in
% \OO^+(J,x)$, $(\rup{A_f} \vec a) (y=b) = a_{yb}$, and for $g \colon
% K \to L$ defined on $J,x$, $(\rup{A_f} \vec a)g = \rup{A_{fg}} (\vec
% a g)$.  Similar, we require operations $\rdo{A_f} \vec a$ for
% downwards open boxes $\vec a$ in $A_f$.  The restrictions of a
% family $A$ in $U(I)$ by $f \colon I \to J$ is given by $(Af)_g =
% A_{fg}$.  This defines $U$ as a cubical set.

Note that the points of $U$ are simply the (small) uniform Kan cubical
sets. More precisely, since $\es$ is initial in $\CC$,
any $A$ in $U(\es)$ becomes a cubical set when we define
$A(I)$ as $A_f$ for the unique $f:\es\to I$.
 A line in $U$ between points $A$ and $B$ can be seen as a
``heterogeneous'' notion of lines, cubes, $\dots$ $a\to b$
where $a$ is an $I$-cube of $A$ and $b$ an $I$-cube of $B$.

As a first step towards proving that this cubical set satisfies the
Kan condition we show how to compose an $A$ and $B$ in $U(I)$ with $x
\in I$ assuming $A (x=1) = B(x=0)$; we define $C = \comp(A,B) \in U
(I)$ such that $C (x=0) = A (x=0)$, $C(x=1) = B(x=1)$, and for $f
\colon I \to J$ defined on $x$, $C f = \comp (Af, Bf)$.  (Compare this
to the composition of relations.)

We define the sets $C_f$, $f \colon I \to J$ by case distinction on $f
x$; in case $f x = 0$, we can write $f = (x=0) f'$ and we have to set
$C_f = A_f$ as we have to satisfy $C_f = (C(x=0))_{f'} = (A
(x=0))_{f'} = A_f$; similarly, if $f x = 1$, we set $C_f = B_f$.  In
case, $f$ is defined on $x$, an element of $C_f$ is any pair $(a,b)$
such that $a \in A_f$ and $b \in B_f$ with $a (x=1) = b (x=0)$.

We still have to define the restrictions $C_f \to C_{fg}$ for $g
\colon J \to K$; in the first two cases from above, the restrictions
are induced by $A_f$ and $B_f$ respectively.  In case $f$ is defined
on $x$, we look at $g(f x)$: if $g (f x)= 0$, we set $(a,b) g = ag$; if
$g (f x) = 1$, we set $(a,b)g = bg$; and if $g$ is defined $f x$, we
define $(a,b)g = (ag,bg)$.

It remains to define the Kan fillings for $C$; it suffices to give
them for $C_1$ as $C_f$ is either determined by $A_f$, $B_f$, or
$\comp (Af,Bf)_1$; so let $J, x' \subseteq I$, $x' \notin J$, and
$\vec u$ be a open box in $C_1$, i.e., $u_{yc} \in C_{(y=c)}$ for
$(y,c) \in \OO^+ (J,x')$ with $u_{yc} (z=d) = u_{zd} (y=c)$.  Note
that for $y \neq x$, $u_{yc}= (a_{yc}, b_{yc})$ with $a_{yc} \in A_1$
and $b_{yc} \in B_1$ with $a_{yc} (x=1) = b_{yc} (x=0)$.  We want to
define $u = \rup{C_1}\vec u$.  There are three cases.  First, in case
$x = x'$, we set $a_{x0} = u_{x0} \in C_{(x=0)} = A_{(x=0)}$; this
yields an open box $\vec a$ in $A_1$ which we can fill to $a =
\rup{A_1} \vec a \in A_1$.  Now setting $b_{x0} = a (x=1)$ yields an
open box $\vec b$ in $B_1$ which we can fill to get $b = \rup{B_1}
\vec b \in B_1$. Note that $b(x=0) = a (x=1)$ and thus we can set $u =
(a, b)$.

Second, in case $x \neq x'$ with $x \in J$, we construct an element $v
\in A_{(x=1)} = B_{(x=0)}$ first.  For $(y,c) \in \OO^+ (J-x, x')$
define $v_{yc} = a_{yc} (x=1)$ (which is also equal to $b_{yc}
(x=0)$).  It is readily checked that this defines an open box in
$A_{(x=1)} = B_{(x=0)}$ and thus we get $v = \rup{A_{(x=1)}} \vec v$.
Now set $a_{x1} = b_{x1} = v$; this yields open boxes $\vec a$ and
$\vec b$ in $A_1$ and $B_1$, respectively.  Thus we can take $u =
(\rup{A_1} \vec a,\rup{B_1} \vec b)$.

Finally, in case $x \notin J$, we directly have open boxes $\vec a$
and $\vec b$ in $A_1$ and $B_1$, respectively. Setting $u = (\rup{A_1}
\vec a,\rup{B_1} \vec b)$ gives an element in $C_1$ since
\[
(\rup{A_1} \vec a) (x=1) = \rup{A_{(x=1)}} (\vec a (x=1)) = \rup
{B_{(x=0)}} (\vec b (x=0)) = (\rup{B_1} \vec b) (x=0).
\]
This concludes the definition of $C=\comp(A,B)$.

\subsection{Equivalence and equality of types}

 We explain in this section how to transform any equivalence $\sigma:A\to B$ between two Kan cubical sets
to an equality path $A\to B$, as defined in the previous section.
By definition (cf.\ \cite[Definition 4.4.1]{HoTTbook}, $\sigma:A\to B$
is an equivalence if there is a map $\delta:B\to A$ and a map
$\sigma\delta b\to b$ and a transformation of any equality
$\omega:\sigma a\to b$, where $a$ (resp. $b$) is an $I$-cube of $A$ (resp. $B$) to a ``square'' (really
a pair of an $I,x$-cube of $A$ and an $I,x,y$-cube of $B$)

\begin{diagram}
a          & \rTo & \delta b    \\
\sigma a   & \rTo & \sigma \delta b \\
\dTo       &      & \dTo           \\
b          & \rTo & b
\end{diagram}

 We define from this a path $C$ between $A$ and $B$ in the direction $x$. For any substitution $f:\{x\}\to I$
we have to define a set $C_f$ together with substitution maps $C_f\to C_{fg}$.
If $f(x) = 0$ we take $C_f = A(I)$ and if $f(x) = 1$ we take $C_f = B(I)$. If $f(x) = y$ then we define $C_f$
to be the set of pairs $(a,b)$ where $a$ is an $(I-y)$-cube of $A$ and $b$ is an $I$-cube of $B$ and
$b(y=0) = \sigma a$.
It can be then be checked in an elementary way that if $\sigma$ is an equivalence,
then this ``heterogeneous'' notion of cube has the uniform Kan property.

 In pictures, the main issue is to complete an open box

\begin{diagram}
\sigma a_0 & \rTo & b_0    \\
           &      & \dTo \\
\sigma a_1 & \rTo & b_1
\end{diagram}

to a square

\begin{diagram}
a_0  & \sigma a_0 & \rTo & b_0    \\
\dTo & \dTo       &      & \dTo \\
a_1  & \sigma a_1 & \rTo & b_1
\end{diagram}

 For this, using the fact that $\sigma$ is an equivalence, we transform the open box in an open box in $A$

\begin{diagram}
a_0 & \rTo & \delta b_0    \\
           &      & \dTo \\
a_1 & \rTo & \delta b_1
\end{diagram}

and since $A$ is Kan, it can be filled to a box

\begin{diagram}
a_0 & \rTo & \delta b_0    \\
\dTo       &      & \dTo \\
a_1 & \rTo & \delta b_1
\end{diagram}

and we can then fill the box in $B$

\begin{diagram}[tight,width=2em,height=2em]
a_0 & & \sigma a_0 &                 &               &  \rTo   &      &           &               b_0 \\
    & &  & \rdTo           &               &               &      & \ldTo     &   \\
    & &  &                 & \sigma\delta b_0      & \rTo          & b_0    &           &   \\
\dTo^{} & & \dTo^{} &        & \dTo           &              & \dTo_{}  &    & \dTo_{}  \\
   & &   &                 & \sigma\delta b_1       & \rTo^{}      & b_1    &           &   \\
 &  &    & \ruTo^{}  &               &               &      &  \luTo    &   \\
a_1 &  & \sigma a_1    &                 &   \rTo        &               &      &                & b_1
\end{diagram}

\medskip

 Since our model is constructive, this gives a way to effectively transport properties and structures
on a Kan cubical set to one which is equivalent. In particular we can effectively transport properties
and structures of a groupoid to one which is categorically equivalent.

\medskip

 We have only checked a weak corollary of the Axiom of Univalence, but we expect the complete Axiom to be valid
in this model in an elementary way as well.


\subsection{Propositional reflection}

 We can describe the operation of Kan ``completion''. Given a cubical set $X$ we add
operations $X^+,X\uparrow,X^-,X\downarrow$ in a {\em free} way, i.e.
considering these operations as {\em constructors}.
 The uniformity condition defines what should be the degeneracies of these elements
We get in this way a new cubical set $Y$, satisfying by definition the Kan extension
property, with a map $X\to Y$. Furthermore, if $Z$ is Kan, and we have a map
$\sigma:X\to Z$ there is a map $Y\to Z$ extending $\sigma$.
This map is furthermore unique if we impose
it to commute with the Kan  operations.
In general however, the maps of Kan cubical sets do not need to commute with the Kan operations.

 The same idea can be used to define the spheres. For instance $\Sph$ will be the
Kan completion of the cubical set generated by a point ${\sf base}$ and a loop ${\sf loop}$.
If we have a dependent type $\Sph\vdash P$ and $a$ in $P~{\sf base}$ and $l$ in $P~{\sf loop}$
then we can define a section $\Sph\vdash t:P$ satisfying
$$
t~{\sf base} = a~~~~~~~~~~~t~{\sf loop} = l
$$

 Finally, we can define $\inh~X$, the {\em proposition} stating that $X$ is inhabited.
For this we add a constructor $\alpha_x(a_0,a_1)$ connecting formally along the dimension $x$
any two $I$-cubes  $a_0$ and $a_1$ (with $x$ not in $I$) and constructors for the Kan filling
and composition operations. We define then
$$\alpha_x(a_0,a_1)(x=0) ~=~ a_0~~~~~~~~\alpha_x(a_0,a_1)(x=1) ~=~ a_1$$
and, if $f$ is defined on $x$ with $y=f(x)$
$$\alpha_x(a_0,a_1)f ~=~ \alpha_y(a_0(f-x),a_1(f-x))$$
 This satisfies the required induction principle of $\inh~X$: if $Y$ is a proposition
and $X\to Y$ then we have $\inh~X\to Y$.

 We can then define $\exists~A~B$ to be $\inh(\Sigma~A~B)$. If $\Sigma~A~B$ is a proposition we have
an inhabitant of $\exists~A~B\to \Sigma~ A~B$ and this can be seen as a generalisation of
the {\em axiom of description} since if $A$ set, $B$ proposition and $B$ is satisfied by at most
one element of $A$ then $\Sigma~A~B$ is a proposition.

\section*{Acknowledgement}
The research for this paper has been started while the first two
authors were members of the Institute for Advanced Study in Princeton,
as part of the program \emph{Univalent Foundations of Mathematics}.
We are grateful for the generous support by the IAS and the Fund for Math.

The last two authors acknowledge financial support from the ERC:  The research leading
to these results has received funding from the European Research Council under
the European Union's Seventh Framework Programme (FP7/2007-2013) / ERC
grant agreement nr.\ 247219.

 The authors wish to thank Jean-Philippe Bernardy, Cyril Cohen, Andy Pitts and Michael Schulman
for stimulating discussions on the topic of this paper. The clear presentation of \cite{Williamson}
provided an important help.


\begin{thebibliography}{9}

\bibitem{AwodeyWarren}
S. Awodey and M. Warren.
\newblock{Homotopy theoretic models of identity types.}
\newblock{Mathematical Proceedings of the Cambridge Philosophical Society, 146, p 45--55, 2009.}

\bibitem{BCH}
B.Barras, Th. Coquand and S. Huber.
\newblock{A generalization of Takeuti-Gandy interpretation.}
\newblock{Preprint, 2013.}

\bibitem{BC}
M.Bezem and Th. Coquand.
\newblock{A Kripke model for simplicial sets.}
\newblock{Preprint, 2013.}

\bibitem{Bishop}
E. Bishop.
\newblock{{\em Foundations of Constructive Analysis.}}
\newblock{New York Academic Press. 1967}


\bibitem{Cartmell}
J. Cartmell.
\newblock{Generalised algebraic theories and contextual categories.}
\newblock{Ann. Pure Appl. Logic 32 (1986), no. 3, 209--243.}

\bibitem{Crans}
S. Crans.
\newblock{On combinatorial models for higher dimensional homotopies.}
\newblock{PhD thesis, 1995.}

\bibitem{Curien}
P.L. Curien.
\newblock{Substitutions up to isomorphisms.}
\newblock{Fundamenta Informaticae, Volume 19, 1993, p. 51-85.}

\bibitem{Dybjer}
P. Dybjer.
\newblock{Internal Type Theory.}
\newblock{in {\em Types for Programs and Proofs}, Springer, 1996.}

\bibitem{Gandy}
R. Gandy.
\newblock{On The Axiom of Extensionality -Part I.}
\newblock{The Journal of Symbolic Logic, Vol. 21, 1956.}

\bibitem{Grothendieck}
A. Grothendieck.
\newblock{Pursuing stacks.}
\newblock{Manuscript, 1983.}

\bibitem{Hofmann}
M. Hofmann.
\newblock{Extensional concepts in intensional type theory.}
\newblock{Ph.D. thesis, Edinburgh, 1994.}

\bibitem{hurewicz}
W. Hurewicz.
\newblock{On the concept of fiber space.}
\newblock{Proc. Nat. Acad. Sci. U.S.A. 41, p. 956--961, 1955.}

\bibitem{Kan}
D. Kan.
\newblock{Abstract homotopy. I.}
\newblock{Proc. Nat. Acad. Sci. U.S.A. 41, p. 1092--1096, 1955.}

\bibitem{MLTT}
P. Martin-L\"of.
\newblock{{An intuitionistic theory of types: predicative part.}}
\newblock{Logic Colloquium, 1973.}

\bibitem{pitts}
A. M. Pitts.
\newblock{{\rm Nominal Sets. Names and Symmetry in Computer Science.}}
\newblock{Cambridge Tracts in Theoretical Computer Science, 2013.}

\bibitem{Pitts}
A. M. Pitts.
\newblock{An Equivalent Presentation of the Bezem-Coquand-Huber
Category of Cubical Sets}. Manuscript, 17 September 2013.

\bibitem{Russell}
B. Russell.
\newblock{{\em Principia Mathematica}, second edition, Introduction.}
\newblock{Cambridge University Press, 1925.}

\bibitem{Russell1}
B. Russell.
\newblock{The Theory of Implications.}
\newblock{American Journal of Mathematics, Vol. 28, 2, p. 159-202, 1906.}

\bibitem{Serre}
J.P. Serre.
\newblock{Homologie simguli\`ere des espaces fibr\'es. Applications.}
\newblock{Th\`ese, Paris, 1951.}

\bibitem{Stoughton}
A. Stoughton.
\newblock{Substitution revisited.}
\newblock{Theoretical Computer Science, Vol. 59, p. 317-325, 1988.}

\bibitem{Takeuti}
G. Takeuti.
\newblock{On a generalized logic calculus.}
\newblock{{\em Japanese Journal of Mathemathics} 23, p. 39-96, 1953.}

\bibitem{Tarski}
A. Tarski.
\newblock{\"Uber die Beschr\"anktheit der Ausdrucksmittel deduktiver Theorien.}
\newblock{In {\em Ergebnisse eines mathematischen Kolloquiums}, facsicule 7 (1934-35),
          English translation ``On the limitations of the means of expression of deductive theories''.}

\bibitem{HoTTbook}
The Univalent Foundations Program.
\newblock{Homotopy Type Theory.}
\newblock{Institute for Advanced Study, 2013.}

\bibitem{Voevodsky}
V. Voevodsky.
\newblock{Univalent foundations project.}
\newblock{NSF grant application, 2010.}

\bibitem{Williamson}
R. Williamson.
\newblock{Combinatorial homotopy theory.}
\newblock{Preprint, 2012.}


\end{thebibliography}


\section*{Appendix 1: Combinatorial definition of $\pi_n(X,x)$}

%In Kan's original definition \cite{Kan} of cubical set the objects of the
%base category are finite linear orders of directions,
%whereas the objects of our base category $\CC$ are finite sets.
Our cubical sets do not have \emph{connections}, like those in \cite{Williamson}.
In \cite[III.4.2.10]{Williamson}, Williamson raises the question of a straightforward
geometric argument replacing his footnote~19 to exhibit the group structure of
$\pi_n(X,x)$ for Kan cubical sets \emph{without} connections.
Adapting his geometric argument,
we can answer his question positively for our Kan cubical sets.
The argument below seems to work also
in the original setting of linearly ordered directions.

 We first explore some consequences of the Kan filling property for a cubical set $A$.
 (We shall not need the uniformity condition.) Using the Kan filling property,
we can complete any equality proof $\omega:a\rightarrow u$ to a square

\begin{diagram}
a & \rTo^{\omega} & u \\
\dTo^{\omega'} & & \dTo         \\
u & \rTo & u
\end{diagram}

 There is no reason for $\omega'$ to be the same as $\omega$.
 We show how to use the Kan property to find such a square where $\omega'$ coincides
with $\omega$, the `inner' or ground square in the following diagram.

\begin{diagram}[tight,width=2em,height=2em]
u    &                 &  \rTo         &               &      &           & u \\
     & \luTo^{\omega'} &               &               &      & \ruTo     &   \\
     &                 &    a          & \rTo^{\omega} & u    &           &   \\
\dTo &                 & \dTo^{\omega} &               & \dTo &           & \dTo  \\
     &                 &     u         & \rTo          & u    &           &   \\
     &  \ldTo          &               &               &      &  \rdTo    &   \\
u    &                 &   \rTo        &               &      &           & u
\end{diagram}

Similarly we show that any equality proof $\omega:a\rightarrow u$ can be completed to a square

\begin{diagram}
a & \rTo & a \\
\dTo & & \dTo_{\omega}         \\
a & \rTo^{\omega} & u
\end{diagram}

 As an application, we can go back and forth between the two squares

\begin{diagram}
a    & \rTo^{\alpha}     & b             & & &    a    & \rTo^{\alpha}  & b \\
\dTo^{\gamma} &          & \dTo_{\beta}  & & &    \dTo &                & \dTo_{\beta}  \\
c & \rTo^{}              & c             & & &    a    &  \rTo^{\gamma} & c
\end{diagram}

using the cube

\begin{diagram}[tight,width=2em,height=2em]
a    &                 &               &  \rTo^{\alpha}&      &           & b \\
     & \luTo           &               &               &      & \ruTo     &   \\
     &                 &    a          & \rTo^{\alpha} & b    &           &   \\
\dTo^{\gamma} &        & \dTo           &              & \dTo_{\beta}&    & \dTo_{\beta}  \\
     &                 &     a         & \rTo^{\gamma} & c    &           &   \\
     & \ldTo^{\gamma}  &               &               &      &  \rdTo    &   \\
c    &                 &   \rTo        &               &      &           & c
\end{diagram}

 Using these remarks, we can define $\pi_1(X,a)$ as follows. The elements are homotopy equivalence classes
of paths $a\rightarrow a$ and two paths $\omega,\omega':a\rightarrow b$ are equivalent iff we can find a square

\begin{diagram}
a & \rTo^{\omega} & b \\
\dTo & & \dTo         \\
a & \rTo^{\omega'} & b
\end{diagram}

 Using the Kan condition, one can show that this is an equivalence relation \cite{Williamson}.

 We define then the composition $\alpha\beta$ of two paths $\alpha:a\rightarrow b,\beta:b\rightarrow c$
as being the path obtained by the Kan filling property

\begin{diagram}
a    & \rTo^{\alpha}     & b             &\\
\dTo &          & \dTo_{\beta}  &\\
a & \rTo_{\alpha\beta}   & c             &
\end{diagram}

The following diagram shows that composition preserves homotopy equivalence

\begin{diagram}[tight,width=2em,height=2em]
a     &                 & &  \rTo^{\alpha'}         &               &         &  b \\
     & \luTo  &               &               &      & \ruTo     &   \\
     &                 &    a          & \rTo^{\alpha} & b    &           &   \\
\dTo &                 & \dTo &               & \dTo_{\beta} &           & \dTo_{\beta'}  \\
     &                 &     a         & \rTo_{\alpha\beta}           & c    &           &   \\
     &  \ldTo          &               &               &      &  \rdTo    &   \\
a    &                 &  & \rTo_{\alpha'\beta'}         &                 &           & c
\end{diagram}

The following diagram shows that composition satisfies associativity

\begin{diagram}[tight,width=2em,height=2em]
a     &                 & &  \rTo^{\alpha}         &               &         &  b \\
     & \luTo  &               &               &      & \ruTo     &   \\
     &                 &    a          & \rTo^{\alpha} & b    &           &   \\
\dTo &                 & \dTo &               & \dTo_{\beta} &           & \dTo_{\eta}  \\
     &                 &     a         & \rTo_{\gamma}           & c    &           &   \\
     &  \ldTo          &               &               &      &  \rdTo^{\delta}    &   \\
a    &                 &  & \rTo_{\epsilon}         &                 &           & d
\end{diagram}


 It is then possible to show in a purely combinatorial way \cite{Williamson} that this defines a group.
The unit  is the homotopy equivalence class of $a\to a$.

 Since it is clear how to define combinatorial the loop space $\Omega(X,a)$ we get in this way a simple
combinatorial definition of $\pi_2(X,a) = \pi_1(\Omega(X,a),1_a),\dots$

\end{document}


\section*{Appendix 2: unfinished experimental stuff}

\takeout{%
\subsection*{The classical Kan condition}
Using classical logic we can prove that the usual Kan condition follows from ours
(even without using uniformity). We start with a diagram of the case where $x,I,M = x,y,z$.

\[
\xymatrix@R=.2cm@C=.2cm{
\ar@{.>}[rrrrrr]&&&&&&&\\
&&&&&&&\\
&&\ar[lluu]\ar[rr]                                                                        &&  \ar[rruu]&  &&\\
\ar[uuu]&y=0  & &x=0&&y=1&\ar[uuu] &\\
&&&&&&&\\
&  &\ar[lluu]\ar[rr] \ar[uuu]                                                                        &&  \ar[rruu]\ar[uuu] & &&\\
\\
\ar@{->}[rrrrrr]&&&&&&&\\
&&&z=1&&&&\\
&  &\ar[lluu]\ar[rr]                                                                        &&  \ar[rruu]& &&\\
\ar@{.>}[uuu]\ar@{->}[rrrrrr]&& &&&&\ar@{.>}[uuu] &\\
&&&z=0&&&&\\
&  &\ar[lluu]\ar[rr] \ar@{.>}[uuu]                                                                        &&  \ar[rruu]\ar@{.>}[uuu] & &&\\
}
\]
}%end\takeout
%\subsection*{Different definitions of the cubical unit interval}

\newcommand{\mycol}[2]{{\tiny\begin{array}{c}{#1}\\{#2}\end{array}}}
\def\oooo{\mycol{0}{0}\!\!\!\!\!\mycol{0}{0}}
\def\llll{\mycol{1}{1}\!\mycol{1}{1}}
\def\ooll{\mycol{0}{0}\!\mycol{1}{1}}
\def\olol{\mycol{1}{0}\!\mycol{1}{0}}


Any set $S$ can be viewed as a cubical set (`nerve') $N(S)$ of
$S$-labelled binary cubes in the following way.
For any object $\set{x_1,\ldots,x_n}$ of $\CC$, the set $S(\set{x_1,\ldots,x_n})$
consists of $S$-valued functions $\alpha(x_1,\ldots,x_n)$ of boolean arguments $x_i$.
(Being a set, $\set{x_1,\ldots,x_n}$ is not ordered. Hence it would be slightly better, albeit cumbersome,
to take $\alpha: \set{0,1}^{\set{x_1,\ldots,x_n}} \to S$.)

\medskip

Maps for $N(S)$ are defined by composition with morphisms in $\CC$ in the following way:
if $m:\set{x_1,\ldots,x_n} \to \set{y_1,\ldots,y_m}$,
then $N(S)(m): \alpha(x_1,\ldots,x_n) \lto \beta(y_1,\ldots,y_m)$,
where $\beta(y_1,\ldots,y_m) = \alpha(m(x_1),\ldots,m(x_n))$.
Here $m(x_i)$ is either the boolean that $m$ assigns to $x_i$,
or the boolean $y_j$ for the unique $j$ such that $m(x_i)=y_j$.
(This is well-defined by the definition of morphism in $\CC$.)

\medskip

Clearly, nerves $N(S)$ are Kan. Define $B=N(\set{0,1})$.

\medskip

There are various (equivalent) ways to define the unit interval $I$ as a cubical set:
\begin{enumerate}
\item The `official'? definition $Hom_\CC(\set{i},\_)$ for some designated variable $i$:
\[
\begin{array}{lcl}
I(\emptyset)&=&\set{(i=0),(i=1)}\\
I(\set{x})&=&\set{(i=0)(x),(i=1)(x),(i=x)}\quad\text{(the last is not a swap but a substitution)}\\
I(\set{x,y})&=&\set{(i=0)(x)(y),(i=1)(x)(y),(i=x)(y),(i=y)(x)}\\
\ldots
\end{array}
\]
The maps between such $Hom$-sets are defined by composition with corresponding morphisms in $\CC$.

\item The `concrete' definition:
\[
\begin{array}{lcl}
I(\emptyset)&=&\set{0,1}\\
I(\set{x})&=&\set{00,01,11}\\
I(\set{x,y})&=&\set{\oooo,\ooll,\olol,\llll}\\
\ldots
\end{array}
\]
Face maps and degeneracies can also be viewed in a `concrete' way.

\item The `set theoretic' definition: let $\set{x_1,\ldots,x_n}$ be an object of $\CC$;
then $I(\set{x_1,\ldots,x_n})$ consists of all boolean functions $\alpha(x_1,\ldots,x_n)$
that are either constant, or the projection on one argument.
%(Better: take $\alpha$ as a function $\set{0,1}^{\set{x_1,\ldots,x_n}} \to \set{0,1}$.)

Maps are defined by composition with morphisms in $\CC$ as above for nerves.
This indeed preserves being constant or a projection.

\end{enumerate}
I have some hope that $B^I$ is not Kan. Let's spell out some details of the attempt.

\medskip
Two important maps (natural transformations) $I\to B$ are:
\def\emb{se}
\def\crs{ce}
\begin{enumerate}
\item $\emb$ is the `straight' embedding $I\to B$,
in particular $\emb(0)=0$, $\emb(1)=1$;%, $\emb(01)=01$;
\item $\crs$ is the `crossed' embedding $I\to B$,
in particular $\crs(0)=1$, $\crs(1)=0$.%, $\crs(01)=10$.
\end{enumerate}

By Yoneda and cartesian closure we have (with $\_\to\_$ denoting sets of maps):
\[
B^I(\set{x_1,\ldots,x_n}) \cong (Hom_\CC(\set{x_1,\ldots,x_n},\_)\times I)\to B
\]
We easily get $\emb,\crs: I\to B$ as \emph{points} of $B^I$, that is,
elements of $B^I(\set{})$. For this we note that $\set{}$ is initial in $\CC$,
such that any $Hom_\CC(\set{},\set{y_1,\ldots,y_m})$ is a singleton.

The next step is that we construct a line from $\emb$ to $\crs$ in $B^I(\set{x})$.
Note first that $B^I(\set{x}) \cong (Hom_\CC(\set{x},\_)\times I)\to B)$
and that $Hom_\CC(\set{x},\_) \cong I$.
Define a map that takes the form $(\alpha,\beta)\lto\gamma$,
for any object $\set{x_1,\ldots,x_n}$ of $\CC$, by:
\[
\gamma(x_1,\ldots,x_n) = \left\{
\begin{array}{ll}
\beta(x_1,\ldots,x_n)&\text{if~}\alpha(x_1,\ldots,x_n)=0,\\
1-\beta(x_1,\ldots,x_n)&\text{otherwise},
\end{array}\right.
\]
where $\alpha,\beta\in I(\set{x_1,\ldots,x_n})$ and $\gamma\in B(\set{x_1,\ldots,x_n})$.
Indeed we get $(0,b)\lto b $ and $(1,b)\lto1-b$. In general, if $\alpha(x_1,\ldots,x_n)=0$,
we get $se$, and if $\alpha(x_1,\ldots,x_n)=1$, we get $ce$.
Hence this is a line from $\emb$ to $\crs$ which we call $sc\in B^I(\set{x})$.
\medskip
Consider the following open box, the hope is that it cannot be filled in
$B^I(\set{x,y}) \cong (Hom_\CC(\set{x,y},\_)\times I)\to B$.
\[
\xymatrix@R=.6cm@C=.6cm{
se\ar@{.>}[rr]                 &&ce                              \\
                    &&                     &\\
se\ar[uu]^{se(x)}\ar[rr]_{se(y) }   &&se\ar[uu]_{sc}                       \\
}
\]

Here are some (equivalent) ways to define the unit square $Q=Hom_\CC(\set{x,y},\_)$:
\begin{enumerate}
\item The `combinatorial' definition:
\[
\begin{array}{lclcl}
Q(\emptyset)&=&\set{(x=0)(y=0),\ldots,(x=1)(y=1)}=\set{(0,0),(0,1),(1,0),(1,1)}\\
Q(\set{z})&=&Q(\emptyset)(z)\cup\set{(x=0)(y=z),\ldots,(y=1)(x=z)}\quad\text{(eight lines)}\\
Q(\set{z,u})&=&Q(\set{z})(u)\cup \set{(x=0)(y=u)(z),\ldots,(y=1)(x=u)(z)}\cup\set{(x,y=z,u),(x,y=u,z)}\\
\ldots
\end{array}
\]
Note that $Q(\set{z,u})$ has two non-degenated cubes. In general there are $4+4n+n(n-1)$ elements
in $Q(\set{x_1,\ldots,x_n})$.

\item The `set theoretic' definition: let $\set{x_1,\ldots,x_n}$ be an object of $\CC$;
then $Q(\set{x_1,\ldots,x_n})$ consists of all pairs boolean functions
$\alpha_b(x_1,\ldots,x_n),~b=0,1$,
that are either both constant, or one constant and the other a projection, or both projections
but then on different arguments.
Maps are defined by composition with morphisms in $\CC$ as above for nerves.
This indeed preserves the definition.
\end{enumerate}

Define a map that takes the form $(\alpha_b,\beta)\lto\gamma$,
for any object $\set{x_1,\ldots,x_n}$ of $\CC$, by:
\[
\gamma(x_1,\ldots,x_n) = \left\{
\begin{array}{ll}
1-\beta(x_1,\ldots,x_n)&\text{if~}\alpha_0(x_1,\ldots,x_n)=\alpha_1(x_1,\ldots,x_n)=1,\\
\beta(x_1,\ldots,x_n)&\text{otherwise}
\end{array}\right.
\]
If $\alpha_b,~b=0,1$ are both constant 1, we get $ce$. In all other cases of constant
$\alpha_b$'s we get $se$. This means the corners are as in the square above.
Edges are obtained by taking one of the $\alpha_b$'s constant.
This shows there is something wrong with this square.
In order to prove that $B^I$ is not Kan, we have to prove that there
does not exist a filling square as above in $B^I(\set{x,y})$.

\medskip
This is how far I came.




